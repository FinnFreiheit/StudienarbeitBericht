%!TEX root = ../../main.tex

\section{Motivation} \label{se:Motivation}

Folgendes Szenario soll einen Einblick in die Vorteile von \ac{PWAs} geben. 

Ein junges Startup aus IT-Studenten hat eine Idee für eine Applikation und möchte diese umsetzen. Ihr Ziel ist es, diese Applikation an so viele Nutzer wie möglich zu verbreiten. Die Applikation soll demnach für folgende Plattformen verfügbar sein - siehe Tabelle \ref{plattformen}.

\begin{table}[!htb]
\begin{tabularx}{\textwidth}{|X|X|}
    \hline
    \textbf{Plattform} & \textbf{Programmiersprache} \\
    \hline
    \hline
    IOS & Swift \\
    \hline
    Android & Java oder Kotlin\\
    \hline
    MacOS & Swift \\
    \hline 
    native Windows & C++ oder C\# \\
    \hline
    Webbrowser & JavaScript, HTML und CSS \\
    \hline
\end{tabularx}
\caption{Plattformen und die dazu benötigten Programmiersprachen}
\label{plattformen}
\end{table}

Wie man anhand der Tabelle sehen kann, kommt eine Vielzahl an unterschiedlichen Programmiersprachen zum Einsatz, um die Applikation über mehrere Plattformen zu verbreiten. Jedoch verfügt das junge Startup nicht über die benötigten Kapazitäten, um die Applikation in jeder dieser Programmiersprachen implementieren und warten zu können. 

Damit die Applikation dennoch an möglichst viele Kunden gelangt, sucht das Startup nach einer anderen Lösung, um die aufwendige Implementierung zu umgehen. Dabei kommt die Idee auf, eine \ac{PWA} zu entwickeln. Eine PWA ist eine Webanwendung, die wichtige Merkmale nativer Apps enthält. Neben der Erfüllung der wichtigsten Anforderungen, die an Apps gestellt werden, liegt die Besonderheit darin, dass sie - einmal implementiert - auf sämtlichen Plattformen installiert werden kann. Dadurch ist sie plattformunabhängig und die Verwendung sämtlicher Programmiersprachen kann umgangen werden.

Für die Programmierung einer PWA werden lediglich HTML, CSS und JavaScript benötigt. Die Verwendung von JavaScript macht die Entwicklung von PWAs besonders attraktiv, da die Programmiersprache laut einer Umfrage von Stack Overflow bereits seit neun Jahren in Folge die Beliebteste ist. Im Jahr 2021 haben 64\% der befragten Entwickler angegeben, JavaScript zu verwenden. Im Gegensatz dazu verwendeten nur 35\% Java, 27\% C\# und 5\% Swift \cite{stackoverflow2021}. 


\section{Problemstellung}\label{Problemstellung}

Die These, dass Progressive Web Apps plattformunabhängig und vollumfänglich eingesetzt werden können, soll im Verlauf dieser Arbeit untersucht werden. 

Hierfür wird eine Applikation entwickelt, die die Grundfunktionen einer PWA implementiert. Die Anwendung soll 
 \begin{itemize}
     \item installierbar sein,
     \item auf Geräteschnittstellen zugreifen können,
     \item offline verfügbar sein und
     \item Push-Benachrichtigungen ermöglichen. 
 \end{itemize}
 
Diese vier Charakteristika sind Grundlage für eine PWA, die eine relevante Alternative zu nativen Apps darstellen soll. Anhand dieser muss geprüft werden, wie fortgeschritten die Entwicklung von PWAs im Moment ist. Mithilfe der Ergebnisse soll die Frage beantwortet werden, ob PWAs reif genug sind, um native Anwendungen langfristig von dem Markt zu verdrängen. 

Neben den Möglichkeiten sollen jedoch auch die Grenzen aufgezeigt werden, die bei der Umsetzung von PWAs deutlich werden. Dazu zählen jene, die im Bereich der Funktionalitäten auftreten können, aber auch diejenigen, welche durch Unternehmen gesetzt werden. Dabei wird es vor allem wichtig, zwei der größten Unternehmen - Apple und Google - im Vergleich zu betrachten und herauszufinden, in wie weit die Entwicklung von PWAs unterstützt oder auch behindert wird.

Im Anschluss wird die Anwendung auf jenen Plattformen ausgeführt, die in Tabelle \ref{plattformen} aufgeführt sind. Dabei wird untersucht, ob alle beschriebenen Funktionalitäten vorhanden und voll ausführbar sind. 

\section{Begriffserklärung}\label{se:Begriffserklaerung}

Der Begriff \textit{Progressive Web App} setzt sich aus den Begriffen \textit{Web App} und \textit{Progressive Enhancement} zusammen. Eine Web App (zu deutsch: Webanwendung) ist eine mithilfe von JavaScript, HTML und CSS entwickelte Applikation. Der zweite Begriff entstand durch Steve Champeon im Jahre 2003 in seiner Publikation mit dem Titel \textit{progressive enhancement and the future of web design} \cite{Champeon}.

Unter dem Begriff \textit{Progressive Enhancement} (zu deutsch: Progressive Verbesserung) verbirgt sich das Ziel, Webseiten so zur Verfügung zu stellen, dass jeder Webbrowser in der Lage ist, die grundlegendste Form einer Webseite darzustellen. Hierbei ist es unabhängig davon, über welche Version der Browser oder das Endgerät verfügt. 
Alle zusätzlichen Funktionalitäten, die sich nur unter Verwendung moderner Browser und Endgeräte nutzen lassen, werden erst im Anschluss in Form von Skripten eingebunden. 

Um PWAs nutzen zu können, werden die neusten Funktionen moderner Webbrowser benötigt, darunter \textit{service workers}(Abschitt \ref{sec:ServiceWorker}) und \textit{web app manifests}(Abschnitt \ref{sec:webappmanifest}). 

Google hat das Konzept von PWAs im Jahr 2015 vorgestellt und ist seit dem maßgeblich an der Entwicklung beteiligt. 
Das Ziel bei der Entwicklung von PWAs liegt darin, die Vorteile Nativer Applikationen mit denen der Webanwendung zu kombinieren. 


Native Applikationen beziehungsweise plattformspezifische Applikationen sind sehr funktionsreich und zuverlässig. Weitere Vorteile sind, dass sie : 
\begin{itemize}
    \item  netzwerkunabhängig funktionieren,
    \item  lokale Dateien aus dem Dateisystem lesen und schreiben können, 
    \item  auf Hardwareschnittstellen wie USB und Bluetooth zugreifen können, 
    \item  mit Daten des Geräts interagieren können, wie zum Beispiel Fotos oder aktuell spielende Musik. 
\end{itemize}

Webapplikationen wiederum sind sehr gut erreichbar, sie können verlinkt, über Suchmaschinen gefunden und geteilt werden. 

Mithilfe von PWAs können Applikation erzeugt werden, die: 
\begin{itemize}
    \item installierbar sind, Abschitt \ref{sec:webappmanifest} 
    \item auf Geräteschnittstellen zugreifen können, Abschnitt \ref{sec:Geraeteschnittstelle} 
    \item netzwerkunabhängig funktionieren, Abschitt \ref{sec:ServiceWorker} und Abschnitt \ref{sec:pwaAngular}
    \item Push-Notifications versenden können, Abschitt \ref{sec:ThPushNotifikations}
\end{itemize}

PWAs sind laut Sam Richard und Pete LePage das Beste aus zwei Welten \cite{SamRichard2020}. Mithilfe von progressiver Verbesserung werden die modernen Funktionen von Browsern genutzt, um die Vorteile von plattformspezifischen Anwendungen nutzen zu können. Sind die dafür benötigten Funktionen wie beispielsweise \textit{service workers} nicht vorhanden, können die Grundfunktionen der Anwendung dennoch im Web genutzt werden. 

\section{Aufbau der Arbeit}\label{se:AufbauDerArbeit}
Ziel der Arbeit ist es, eine Progressive Web App zu entwickeln und anhand dieser die Umsetzung und Reife ihrer Charakteristika zu bewerten. 
Dazu müssen zunächst die theoretischen Grundlagen geschaffen werden. Hier gilt es zwischen dem Frontend, dem Backend und den PWA-spezifischen theoretischen Inhalten zu unterscheiden.\\
Für das Frontend wird das Web-Framework Angular verwendet, dessen Eigenschaften und Funktionsweisen zunächst erläutert werden sollen. Das Backend enthält eine MongoDB, in der alle Daten von Userdaten über hochgeladene Bilder bis hin zu Kommentardaten gespeichert werden sollen. Im Laufe der Beschreibung des Aufbaus und der Funktionsweise der Datenbank wird außerdem erläutert, wieso man sich für eine nicht-relationale Datenbank und speziell für MongoDB entschieden hat.\\
Um die Verbindung von Frontend und Backend herzustellen, benötigt man einen Server, der das Routing von Anfragen übernimmt. Dazu kommt Node.js zum Einsatz. Speziell das Modul Express soll beschrieben werden, da über jenes das Routing – teils auch über Middleware – übernommen wird.
Nachdem der aktuelle technische Stand des Aufbaus einer Webpage erläutert wurde, soll im Anschluss darauf eingegangen werden, wie man aus einer normalen Webanwendung eine PWA entwickeln kann. Web App Manifests, Service Workers und Push Notifications sind dabei wichtige Elemente, die beschrieben werden müssen, um alle Grundlagen einer PWA zusammenzufassen. \\
Nach Schaffen der technischen Grundlagen sollen diese praktisch umgesetzt werden. Dafür soll zunächst bestimmt werden, was die Webanwendung können soll. Dazu gehört das Layout der Anwendung, die Optionen, welche man während der Anwendung ausführen kann sowie die Reaktionen des Backends. Um die Antworten des Backends auf die Anfragen des Frontends umzusetzen, müssen die zugehörigen Requests und Callback-Funktionen im Backend erstellt werden. Um die Webanwendung später als eine PWA installieren, das heißt sie offline verwenden und Daten speichern zu können, müssen im Anschluss Service Workers und Web App Manifests eingebunden werden.\\
Nach Fertigstellen der App soll diese auf verschiedenen Geräten und in verschiedenen Browsern installiert und bewertet werden. Ein kritischer Blick muss vor allem auf die Offline-Funktionalitäten, den Zugriff auf Geräteschnittstellen, die Installierbarkeit und die Push-Nachrichten geworfen werden. 
