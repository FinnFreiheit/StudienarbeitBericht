%!TEX root = ../../main.tex
\chapter{Bewertung der Plattformunabhängigkeit}

Apple unterstützt die Entwicklung von PWAs nicht. PWAs stellen  eine Möglichkeit dar, um Apps für das iPhone zu installieren, die sich nicht in dem App Store befinden. Im Jahr 2020 hat Apple mit dem App Store einen Umsatz von 643 Milliarden US-Dollar erwirtschaftet \cite{Kirchenbauer2021}. Apple erhält 30\% Provision, wenn Apps gekauft oder In-App-Käufe abgeschlossen werden. Durch den Einsatz von PWAs könnte diese Einnahmen reduziert werden.

Google ist maßgeblich an der Entwicklung von PWAs beteiligt. Dementsprechend Funktionieren PWAs im Google-Ökosystem am besten. Besonders gut ist die Unterstützung von PWAs bei der Installation zu beobachten. In Google Chrome wird nur ein Klick auf den sehr präsenten Button in der Suchleiste benötigt und unter Android wird den Anwender aktiv gefragt ob er die PWA installieren möchte. Auch sämtliche Funktionen wie Push-Notifications, Caching und Geräteschnittstellen werden vollumfänglich unterstützt. 
Google unterstützt nicht nur die Anwender von PWAs sondern auch die Entwickler. In dem von Google entwickelten Webframework Angular benötigt es nur wenige befehle um eine bereits existierende Angular-Webanwendung in eine PWA zu erweitern. 

Obwohl PWAs nicht vollumfänglich im Apple-Ökosystem verwendet werden können sind wir dennoch davon überzeugt, das sie in Zukunft immer mehr an Bedeutung gewinnen werden. Durch die Entwicklung einer PWA waren wir in der Lage eine Anwendung für den Rechner, Web und das Smartphone zu entwickeln und zu installieren. Als Programmiersprache wurde JavaScript verwendet. JavaScript ist laut der Umfrage von Stack Overflow bereits seit neun Jahren im Folge die beliebteste Programmiersprache. Im Jahr 2021 haben 64\% der befragten Entwickler angegeben JavaScript zu verwenden.
Im Gegensatz verwendeten nur 35\% Java, 27\% C\# und 5\% Swift \cite{stackoverflow2021}. 


