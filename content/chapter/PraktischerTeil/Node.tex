%!TEX root = ../../main.tex
\chapter{Node.js und Node Express}

\section{Routing}
Unter Verwendung von Node Express soll das Routing umgesetzt werden. Zunächst müssen die Endpunkte mitsamt ihrer Callback-Funktionen definiert werden, bevor festgelegt wird, wie man sie verwenden kann.

\subsection{Endpunkte}

Mithilfe des Routings kann auf die Daten innerhalb der MongoDB zugegriffen werden. Dazu werden verschiedene Endpunkte benötigt. Die verwendeten Endpunkte sind in den folgenden Tabellen definiert.

\begin{table}[!htb]
\begin{tabularx}{\textwidth}{|X|X|X|}
    \hline
    \textbf{Method} & \textbf{URL} & \textbf{Description} \\
    \hline
    \hline
    GET & /users & alle User abrufen\\
    \hline
    GET & /users/id/:id & User über ID abrufen\\
    \hline
    GET & /users/id/:id/username & Username eines User über ID abrufen\\
    \hline 
    GET & /users/id/:id/password & Passwort eines User über ID abrufen\\
    \hline
    GET & /users/id/:id/pictureID & PictureID eines User über ID abrufen\\
    \hline
    GET & /users/login & Zusammen mit dem Pfad wird ein JSON-Objekt gesendet, das den Usernamen und das Passwort enthält, die am Login übergeben wurden. Damit wird überprüft, ob ein User mit diesen Daten vorhanden ist.\\
    \hline
    POST & /users & User erstellen\\
    \hline
    PATCH & /users/:id & User über ID updaten\\
    \hline
    DELETE & /users/:id & User über ID löschen\\
    \hline
\end{tabularx}
\caption{User Endpunkte}
\label{endpunkte}
\end{table}

\begin{table}[!htb]
    \begin{tabularx}{\textwidth}{|X|X|X|}
        \hline
        \textbf{Method} & \textbf{URL} & \textbf{Description} \\
        \hline
        \hline

    GET & /comments & alle Kommentare abrufen\\
    \hline
    GET & /comments/:id & Kommentar über ID abrufen\\
    \hline
    GET & /comments/:id/username & Usernamen des Users, der den Kommentar geschrieben hat, über ID abrufen\\
    \hline
    GET & /comments/:id/message & Inhalt des Kommentars über ID abrufen\\
    \hline
    POST & /comments & Kommentar erstellen\\
    \hline
    DELETE & /comments/:id & Kommentar über ID erstellen\\
    \hline
\end{tabularx}
\caption{Comments Endpunkte}
\label{commentsendpunkte}
\end{table}

\begin{table}[!htb]
    \begin{tabularx}{\textwidth}{|X|X|X|}
        \hline
        \textbf{Method} & \textbf{URL} & \textbf{Description} \\
        \hline
        \hline
    GET & /posts & alle Posts abrufen\\
    \hline
    GET & /posts/:id & Post über ID abrufen\\
    \hline
    GET & /posts/:id/caption & Unterschrift des Posts über ID abrufen\\
    \hline
    GET & /posts/:id/picture & Bild eines Posts über ID abrufen\\
    \hline
    GET & /posts/:id/geoloc & Geolocation eines Posts über ID abrufen\\
    \hline
    POST & /posts & Post erstellen\\
    \hline
    PATCH & /posts/:id & Post über ID updaten\\
    \hline
    DELETE & /posts/:id & Post über ID löschen\\
    \hline
\end{tabularx}
\caption{Posts Endpunkte}
\label{postsendpunkte}
\end{table}

\begin{table}[!htb]
    \begin{tabularx}{\textwidth}{|X|X|X|}
        \hline
        \textbf{Method} & \textbf{URL} & \textbf{Description} \\
        \hline
        \hline

    GET & /likes & alle Likes abrufen\\
    \hline
    GET & /likes/:id & Like über ID abrufen\\
    \hline
    GET & /likes/:id/userid & ID des Users, der Like gegeben hat, über ID abrufen\\
    \hline
    GET & /likes/:id/pictureid & ID des Bilds, für das der Like gegeben wurde, über ID abrufen\\
    \hline
    POST & /likes & Like erstellen\\
    \hline
    DELETE & /likes/:id & Like über ID löschen\\
    \hline

\end{tabularx}
\caption{Likes Endpunkte}
\label{likes endpunkte}
\end{table}

\clearpage
\subsection{Request-Funktionen}
Innerhalb eines Routing-Files werden alle Responses auf mögliche http-Anfragen des Clients festgelegt. Grundlegend sehen die jeweiligen GET-, POST-, PATCH- und DELETE-Funktionen gleich aus, weshalb im Folgenden zu jeder Anfrage ein Beispiel erläutert wird.

\begin{lstlisting}[caption=GET-Request, label=lst:getrequest, float=!htb]
    router.get('/users/id/:id', async(req, res) => {
        try {
            const user = await User.findOne({ _id: req.params.id });
            console.log(req.params);
            res.send(user);
        } catch {
            res.status(404);
            res.send({
             error: "User does not exist!"
            });
        }
    });
\end{lstlisting}

Mit der obigen Beispiel-Funktion wird ein User über seine ID unter Verwendung der http-Request-Methode GET aus der Datenbank ausgelesen. Der zugehörige Endpunkt ist \texttt{/users/id/:id}. Unter \texttt{:id} kann der Client später den Identifier des Users einfügen, um anhand dessen die Daten dieses Users herauszufinden. Nach dem Pfeil wird die Callback-Funktion definiert. Sie beschreibt, wie der Server auf die Anfrage reagieren soll. Dabei soll er zunächst probieren, den der ID zugeordneten User zu finden. Findet der Server diesen User in der Datenbank, wird er mit allen enthaltenen Daten zurückgegeben. Ist dies nicht der Fall, wird ein Error geworfen und die Konsole enthält die Information \texttt{User does not exist!}.

\begin{lstlisting}[caption=POST-Request, label=lst:postrequest,float=!htb]
    router.post('/users', async(req, res) => {
        const newUser = new User({
            _id: req.params.id,
            username: req.body.username,
            password: req.body.password,
            pictureId: req.body.pictureId,
        })
        await newUser.save();
        res.send(newUser);
    });
\end{lstlisting}

In obigem Code kann man eine POST-Anfrage an den Endpunkt \texttt{/users} erkennen. Im Zuge dieser Anfrage wird ein JSON-Objekt gesendet, dessen Inhalte über \texttt{req.body.*} aufgerufen werden können. Mit den enthaltenen Daten wird ein neuer User in der Datenbank angelegt. Zusätzlich erstellt MongoDB automatisch eine ID und weißt sie dem neuen User zu.

\begin{lstlisting}[caption=PATCH-Request, label=lst:patchrequest,float=!htb]
    router.patch('/users/:id', async(req, res) => {
        try {
            const user = await User.findOne({ _id: req.params.id })

            if (req.body.username) {
                user.username = req.body.username
            }

            if (req.body.password) {
                user.password = req.body.password
            }

            await User.updateOne({ _id: req.params.id }, user);
            res.send(user)
        } catch {
            res.status(404)
            res.send({ error: "User does not exist!" })
        }
    });
\end{lstlisting}

In diesem Beispiel kann man eine PATCH-Anfrage erkennen, mit der die Daten eines Users über seine ID aktualisiert werden können. Mit der Anfrage an den Endpunkt \texttt{/users/:id} wird ebenfalls ein JSON-Objekt gesendet, welches die neuen Daten enthält. Findet der Server in diesem Objekt neue Daten zu einem Schlüssel, wird dessen alter Wert überschrieben. Ist der Server nicht in der Lage einen User mit der eingegebenen ID zu finden, wird ein Error geworfen.

\begin{lstlisting}[caption=DELETE-Request, label=lst:deleterequest,float=!htb]
    router.delete('/users/:id', async(req, res) => {
        try {
            await User.deleteOne({ _id: req.params.id })
            res.status(204).send()
        } catch {
            res.status(404)
            res.send({ error: "User does not exist!" })
        }
    });
\end{lstlisting}

In dem letzten Beispiel wird eine DELETE-Anfrage beschrieben. Mit dieser wird über den Endpunkt \texttt{/users/:id} ein User anhand seiner ID gelöscht. Findet der Server diesen User in der Datenbank, kann er ihn löschen und den Success-http-Status \texttt{204} zurücksenden. Im Fehlerfall wird ein Error geworfen und der Error-Status \texttt{404} gesendet.

\subsection{Login}
Um sich als User einzuloggen, muss man innerhalb des Login-Felds den bei der Registrierung gewählten Usernamen sowie das zugehörige Passwort eingeben. In einem JSON-Objekt werden sowohl Username als auch Passwort an die Datenbank geschickt. Dazu wird folgender API-Request definiert:

\begin{lstlisting}[caption=Login-Request, label=lst:loginrequest,float=!htb]
    router.get('/users/login', async(req, res) => {
        try {
            const user = await User.findOne({ 
                username: req.body.username,
                password: req.body.password,
            });
            console.log('login request');
            res.send(user.id);
        } catch {
            res.status(401);
            res.send({
                error: "Username or password wrong!"
            });
        }
    });
\end{lstlisting}

Den Usernamen sowie das Passwort kann die Datenbank aus dem Body des JSON-Objekts herauslesen. Anhand dieser wird ein User gesucht, der diese Daten enthält. Ist dies der Fall, wird die ID des Users an den Client zurückgesendet, der damit weiterarbeiten kann. Sollte kein User vorhanden sein, der den eingegeben Usernamen und das Passwort enthält, wird der http-Errorstatus \texttt{401} zurückgesendet und der Fehler \texttt{Username or password wrong!} ausgegeben.

\subsection{Verwendung der Endpunkte}
Die Requests und Callback-Funktionen sind bereits definiert. Nun muss der Server angewiesen werden, wie er diese zu verwenden hat. Dazu wird in einer separaten Datei \texttt{server.js} zunächst festgelegt, wo die gebauten Routen zu finden sind. Später können diese Routen mit \texttt{app.use()} aufgerufen werden und sind für den Client verfügbar. Ebenso muss in dieser Datei der Port festgelegt, über den die Endpunkte zu erreichen sind, sowie die Verbindung zu der Datenbank aufgebaut werden.

\begin{lstlisting}[caption=Verwendung der API-Routen, label=lst:apiroutenverwendung,float=!htb] 
    const routes = require('./routes/routes');
    app.use('/', routes);
\end{lstlisting}
