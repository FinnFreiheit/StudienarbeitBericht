%!TEX root = ../../main.tex
\chapter{Layout}

Im Folgendem wird beschrieben, wie die Applikation aufgebaut ist. Und wie durch ein Responsive Design die Anwendung auch unabhängig von der Bildschirmgröße übersichtlich dargestellt werden kann. 

\section{Layout Webanwendung}
Die Abwendung verfügt über folgende von einander Abgrenzbare Elemente: 
\begin{itemize}
    \item obere Navigationsleiste
    \item Hauptfeld
    \item Fußzeile
\end{itemize}

Die obere Navigationsleiste beinhaltet den Title der Anwendung, ein Knopf um Push-Nachrichten zu Abonnieren, und einen um zur Anmeldeseite zu gelangen. Im Hauptfeld werden wahlweise Fotos oder Kommentare angezeigt. Im unterem Teil des Hauptfeldes befinden sich zwei Icons. Das Herz-Icon kann vom Nutzer ausgewählt werden, was zu einer Inkrementierung des Zählers führt, der wiederum in der oberen rechten Ecke des Herz-Icons zu sehen ist. Das Kommentar-Icon dient als Navigationsobjekt um zwischen der Anzeige von Bildern und Kommentaren zu wechseln. Das HTML-Grundgerüst der Anwendung ist in Listing \ref{lst:grundHTML} dargestellt. 

\begin{lstlisting}[caption={HTML-Grundgerüst der Webanwendung}, label=lst:grundHTML]
    <div class="pageContainer">
        <div class="navbar">
            <app-navbar></app-navbar>
        </div>
        <div class="insta">
            <app-insta-container></app-insta-container>
        </div>
        <div class="footer">
        </div>
    </div>
\end{lstlisting}

