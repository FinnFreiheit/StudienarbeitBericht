%!TEX root = ../../main.tex
\chapter{Funktionalitäten}

Startet man die Progressive Web App gelangt man zunächst zum Login. Hier muss man den eigenen Benutzernamen sowie das Passwort eingeben, um weiter zu gelangen. Hat man noch keinen User angelegt, kann man sich neu registrieren. Bei Klicken des \texttt{Create an account}-Buttons wird man auf eine neue Seite geleitet, um dort einen User anzulegen. Dafür muss ein Benutzername und ein Passwort gewählt werden. Nachdem das Passwort wiederholt wurde, kann man sich anmelden und hat einen neuen User registriert.

Auf der Hauptseite befindet sich in der Mitte ein Bild, das von einem User in die PWA hochgeladen wurde. Darunter kann man über das Herz den Post liken sowie die Anzahl der Likes, die das Bild bereits erhalten hat, sehen. Daneben befindet sich eine Sprechblase, über die man den Bereich der Kommentare erreichen kann. Klickt man darauf, wird man zu einer neuen Seite geleitet, auf der die Kommentare anderer User zu diesem Post zu sehen sind. Ebenso kann man unten in einem Schreibfeld selbst einen Kommentar eingeben und über \texttt{send} posten. Klickt man erneut auf die Sprechblase, gelangt man zurück zur Hauptseite. Hier kann man mit den Pfeilen, die sich am linken und rechten Rand des Bildes befinden, zu anderen Posts wechseln und sich neue Bilder anzeigen lassen.

Außerdem kann man rechts oben über den \texttt{Subscribe to push}-Button Push-Nachrichten aktivieren. 

Zuletzt kann man - sofern man angemeldet ist - selbst Posts hochladen und sie anderen anzeigen lassen. Dazu wählt man das Upload-Icon im unteren rechten Rand der PWA aus und gelangt auf eine neue Seite. Dort lässt sich per Drag and Drop ein Bild reinziehen, das im Anschluss gepostet werden kann. 