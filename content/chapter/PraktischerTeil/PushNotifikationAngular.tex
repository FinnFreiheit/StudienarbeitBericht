%!TEX root = ../../main.tex
\chapter{Push-Notifikation}\label{ch:PushNotifikation}

\section{Registrierung}

Damit die Webanwendung Push-Nachrichten empfangen kann, muss sie sich vorab bei dem Push-Dienst registrieren. Die Registrierung erfordert das Einverständnis des Anwenders. 
Um das Anwendererlebnis zu verbessern, wird dieser nicht direkt danach gefragt, sondern kann über den Button \texttt{SUBSCRIBE TO PUSH} die Registrierung anstoßen, siehe Abbildung~\ref{img:permissionPush}. Um das Konzept von Progressive Enhancement einzuhalten, wird der Button nur den Nutzern angezeigt, deren Browser über die nötigen Funktionalitäten verfügt um Push-Nachrichten empfangen zu können. Hierfür wird der Button über eine \texttt{*ngIf}-Strukturdirektive ein- bzw ausgeblendet (Abschitt \ref{sec:Strukturdirektive}), jenachdem, ob der Browser über einen \texttt{PushManager} verfügt oder nicht. 

\begin{figure}[!htb]
    \centering
    \includegraphics[scale=0.4]{permissionPush.png}
    \caption{Nutzerabfrage, um Push-Nachrichten versenden zu können. Abfrage erfolgt erst, nachdem der Button \texttt{SUBSCRIBE TO PUSH} in der oberen Navigationsleiste ausgewählt wurde.}
    \label{img:permissionPush}
\end{figure}

Der Button ist mithilfe eines Event Bindings (Abschitt \ref{subsec:eventBinding}) an die Funktion \linebreak\texttt{subscribeToNotifications()} verknüpft. Da es sich bei der Anwendung um ein Angluar Projekt handelt, 
wird die Registrierung über die \texttt{SwPush}-Klasse durchgeführt (Listing \ref{lst:subscribeFkt} Zeile 2), die mithilfe von dependency injection über den Konstruktor geladen wird. Wie in Abschnitt \ref{sec:pushRegistrirung} beschrieben, wird die Push-Kommunikation durch das VAPID-Protokoll verschlüsselt. Hierfür muss bei der Registrierung der öffentliche Schlüssel übergeben werden (Listing \ref{lst:subscribeFkt} Zeile 3). Anders als in Abschnitt \ref{sec:pushRegistrirung} beschrieben, ist der Standardwert der Eigenschaft \texttt{userVisibleOnly} aus dem Konfigurationsobjekt \texttt{true}. Diese Einstellung wird von der Klasse \texttt{SwPush} vorgenommen. 

Ist die Registrierung erfolgreich, wird ein Objekt von typ \texttt{PushSubscription} zurückgegeben. 
Das Objekt verfügt über folgende Eigenschaften: 
\begin{itemize}
    \item \textbf{endpoint}: Hinter dieser Eigenschaft befindet sich eine einzigartige URL, die für die Kommunikation mit dem Push-Dienst verwendet wird. 
    \item \textbf{expirationTime}: Durch diese Eigenschaft können Nachrichten versendet werden, die nur für einen bestimmten Zeitraum gelten. 
    \item \textbf{keys}: Enthält Informationen, um die Nachrichten zu entschlüsseln. 
\end{itemize}
Dieses Objekt muss im Anschluss über ein HTTP-Request an das Backend gesendet werden (Listing \ref{lst:subscribeFkt} Zeile 7-8). Hierbei wird die Kommunikation mit dem Backend in einem \texttt{Service} ausgelagert. 

Falls die Registrierung erfolglos war, wird in der \texttt{catch}-Methode des Promises der Fehler auf der Konsole ausgegeben (siehe Listing \ref{lst:subscribeFkt} Zeile 11). 

Nach der Registrierung und dem Versenden des \texttt{PushSubscription} Objektes an das Backend müssen keine weiteren Vorkehrungen im Frontend unternommen werden. 

\begin{lstlisting}[caption={Funktion zur Registrierung beim Push-Dienst},label = lst:subscribeFkt,  float=!htb]
subscribeToNotifications(){
    this.swPush.requestSubscription({
      serverPublicKey: this.VAPID_PUBLIC_KEY
    })
      .then(subscribeObj => {
        console.log(subscribeObj);
        this.pushService.addPushSubscriber(subscribeObj).subscribe(res => {
          console.log('Backend Response Push Subscription' + res);
        });
      })
      .catch(err => console.error("Could not subscribe to notifications", err));
  }
\end{lstlisting}

\newpage
\section{Push-Nachrichten im Backend}

Im Backend werden zwei \texttt{post}-Endpunkte implementiert. Der Subscription Endpunkt wird verwendet, um das \texttt{PushSubScription} Objekt entgegenzunehmen. Das Objekt wird in einer Datenbank gespeichert. 

Mit dem zweiten Endpunkt werden verwendet um Push-Notifications angestoßen. Das Backend benutzt die Web-Push-Bibliothek, um allen registrierten Klienten eine Push-Nachricht zu senden.  Die \texttt{sendNotifikation}-Methode der Web-Push-Bibliothek benötigt einen Payload und ein \texttt{PushSubScription}-Objekt. Der Payload enthält die Informationen, die in der Push-Nachricht angezeigt werden, Hierbei muss mindestens ein Title und eine Nachricht vorhanden sein, siehe Listing \ref{lst:payload}. 

\begin{lstlisting}[caption={Mindestanforderung an einen Payload für eine Push-Nachricht}, label=lst:payload, float=!htb]
    const notificationPayload = {
        notification: {
            title: 'Neue Notification',
            body: 'Das ist der Inhalt der Notification'
        },
    }
\end{lstlisting}

Die Web-Push-Methode verschlüsselt die Nachricht vor dem Versenden. Hierfür müssen die VAPID-Schlüsselpaare der Methode \texttt{webpush.setVapidDetails()} übergeben werden. 
Wird nun eine HTTP-Post-Anfrage an diesen Endpunkt gesendet, iteriert die Funktion über alle in der Datenbank gespeicherten \texttt{PushSubscription}-Objekte und sendet über die \texttt{sendNotifikation}-Methode den verschlüsselten Payload. Alle registrierten Klienten erhalten folgende Mitteilung, siehe Abbildung \ref{img:notificationBanner}.

\begin{figure}[!htb]
    \centering
    \includegraphics[scale=0.7]{notificationBanner.png}
    \caption{Push-Nachricht, mit dem Payload aus Listing \ref{lst:payload}}
    \label{img:notificationBanner}
\end{figure}


Apple unterstützt weder im Browser noch im IOS-Betriebssystem die Push-API. Im Browser Safari ist der Push-Manager nicht implementiert, siehe Abbildung \ref{img:PushManagerBrowser}, weshalb  Push-Benachrichtigungen wie unter Google Chrome nicht möglich sind. 

\begin{figure}[!htb]
    \centering
    \includegraphics[scale=0.4]{PushManagerSafari.png}
    \caption{Apple Safari verfügt nicht über einen Push-Manager und ist somit nicht in der Lage, Push-Nachrichten zu erhalten}
    \label{img:PushManagerBrowser}
\end{figure}


Der Browser Safari bietet als Alternative \textit{Safari Push Notifications}. Um Safari Push Notifications versenden zu können, muss das Backend mit dem \ac*{APNs} kommunizieren. Die Kommunikation erfolgt jedoch \textbf{nicht} über das standardisierte Web-Push-Protokoll.

Die APNs Kommunikation erfordert, dass der Entwickler sich bei dem Apple Developer Programm registriert und eine Push-ID für die Webanwendung anlegt. Anschließend muss der Entwickler ein \ac*{TLS}-Clientzertifikat anfordern. In der Safari Umgebung wird anstelle des \texttt{PushSubscription}-Objektes ein \texttt{Device-Token} erzeugt, um Pushbenachrichtigungen zu versenden. 

Es ist somit ein erhöhter Programmieraufwand nötig, um Push-Notifications für Apple-Umgebungen zu realisieren.
