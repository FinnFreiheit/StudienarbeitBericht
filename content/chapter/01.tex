%!TEX root = ../../main.tex

\chapter{Einleitung} \label{ch:Einleitung}
\section{Motivation} \label{se:Motivation}

Folgendes Szenario soll ein Einblick in die Vorteile von \ac{PWAs} aufzeigen. 

Ein junges Startup aus IT-Studenten hat eine Idee für eine Applikation. Ihr Ziel ist es diese Applikation an so viele Nutzer wie möglich zu verbreiten. Die Applikation soll daher für folgende Plattformen, siehe Tabelle \ref{plattformen} erhältlich sein.

\begin{table}[!htb]\label{plattformen}
\begin{tabularx}{\textwidth}{|X|X|}
    \hline
    \textbf{Plattform} & \textbf{Programmiersprache} \\
    \hline
    \hline
    IOS & Swift \\
    \hline
    Android & Java oder Kotlin\\
    \hline
    MacOS & Swift \\
    \hline 
    native Windows & C++ oder C\# \\
    \hline
    Webbrowser & JavaScript, HTML und CSS \\
    \hline
\end{tabularx}
\caption{Plattformen und die dazu benötigten Programmiersprachen}
\end{table}

Wie man anhand der Tabelle sehen kann, wird eine Vielzahl an unterschiedlichen Programmiersprachen benötigt, um die Applikation über mehrere Plattformen zu verbreiten. Das junge Startup verfügt leider nicht über die Kapazitäten um die Applikation in jeder dieser Programmiersprachen zu implementieren und zu warten. 

Aus diesem Grund entscheidet sich das Startup dafür eine \ac{PWA} zu erstellen. Eine PWA ist eine Webanwendung mit erweiterten Funktionen. Die Besonderheit dieser erweiterten Webanwendung liegt darin, das sie einmal implementiert auf sämtliche Plattformen installiert werden kann. Sie ist somit Plattform unabhängig. 

\section{Begriffsklärung}\label{se:Begriffsklaerung}

der Begriff \textit{Progrssive Web App} setzt sich aus den Begriffen \textit{Web App} und \textit{Progressive Enhancement} zusammen. Eine Web App (deutsch Webanwendung) ist eine mithilfe von JavaScript, HTML und CSS entwickelte Applikation. Der zweite Begriff wurde von Steve Champeon im Jahre 2003 in seiner Publikation mit dem Titel \textit{progressive enhancement and the future of web design} geprägt \cite{Champeon}.

Unter dem Begriff \textit{Progressive Enhancement} (deutsch Progressive Verbesserung) verbirgt sich das Ziel Webseiten so zur Verfügung zu stellen, dass jeder Webbrowser in der Lage ist, die grundlegendste Form einer Webseite dazustellen. Hierbei ist es unabhängig über welche Version der Browser oder das Endgerät verfügt. 
Alle zusätzlichen Funktionalitäten, die eventuell erst mit modernen Browsern und Endgeräte genutzt werden können, werden erst im anschluss in form von Skripten eingebunden. 

Um PWAs nutzen zu können werden die neusten Funktionen der modernen Webbrowser benötigt, darunter \textit{service workers} und \textit{web app manifests} (Referenz Kapitel). 

Google hat das Konzept von PWAs im Jahr 2015 vorgestellt und ist seit dem maßgeblich an der Entwicklung beteiligt. 
Das Ziel bei der Entwicklung von PWAs liegt darin die Vorteile von Nativen Applikation mit den Vorteilen von Webanwendung zu kombinieren. 


Native Applikation beziehungsweise Plattformspezifische Applikation sind sehr Funktionsreich und zuverlässig. Weitere Vorteile sind, das sie : 
\begin{itemize}
    \item  Netzwerkunabhängig funktionieren,
    \item  lokale Dateien aus dem Dateisystem lesen und schreiben können, 
    \item  auf Hardwareschnittstellen wie USB und bluetooth zugreifen können, 
    \item  mit Daten des Gerätes interagieren können, wie zum Beispiel Fotos oder aktuell spielende Musik. 
\end{itemize}

Webapplikationen wiederum sind sehr gut erreichbar, sie können verlinkt, über Suchmaschinen gefunden und geteilt werden. 

Mithilfe von PWAs können Applikation erzeugt werden, die: 
\begin{itemize}
    \item Installierbar sind, Kapitel, 
    \item auf Geräteschnittstellen zugreifen können, Kapitel 
    \item Netzwerkunabhängig funktionieren, Kapitel
    \item Push-Notifikations versenden können, Kapitel
\end{itemize}

PWAs sind somit laut Sam Richard und Pete LePage das beste aus zwei Welten \cite{SamRichard2020}. Mithilfe von progressiver Verbesserung werden die modernen Funktionen von Browsern genutzt um die Vorteile von Plattformspezifischen Anwendungen nutzen zu können. Sind die dafür benötigten Funktionen wie zum Beispiel \textit{service workers} nicht vorhanden, können denoch die Grundfunktionen der Anwendung im Web genutz werden. 

\section{Aufbau der Arbeit}\label{se:AufbauDerArbeit}