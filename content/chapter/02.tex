%!TEX root = ../../main.tex
\chapter{Theoretische Grundlagen}

\section{Progressive Web App Grundlagen}

Wie in Kapitel \ref{se:Begriffsklaerung} erwähnt verfügt eine PWA unter anderem über folgende Funktionen: 
\begin{itemize}
    \item Installierbar,
    \item zugriff auf Geräteschnittstellen, 
    \item Netzwerkunabhängig,
    \item Push-Notifikations.
\end{itemize} 

Im folgenden werden die theoretischen Grundlagen erläutert, die benötigt werden, um diese Funktionalitäten zu realisieren. 


\subsection{Web-App-Manifest}

\begin{quote}
   \textit{The web app manifest is a JSON file that defines how the PWA should be treated as an installed application, including the look and feel and basic behavior within the operating system \cite{Developers2022}. }
\end{quote}

Eine PWA kann auf ein Endgerät wie zum Beispiel ein Desktop oder Handy installiert werden. Um diese Funktion zu realisieren, müssen zusätzliche Informationen wie zum Beispiel der Name und das Icon der installierten Applikation in einer Datei festgehalten werden. 

Bei dieser Datei handelt es sich um das sogenannte Web-App-Manifest. Die Informationen sind im \ac{JSON}-Format\footnote{durch Komma getrennte Schlüssel-Wert paare} angegeben. 

Ohne Das Manifest ist die Applikation nicht installierbar, somit ist die Datei eine zwingende Voraussetzung für eine PWA. 
Das Manifest muss mindestens ein \texttt{name}-Schlüssel und ein \texttt{String}-Wert aufweisen. 
Neben dem Namen der Applikation kann ein Manifest über folgende Informationen verfügen: 

\subsubsection{short\_name}
Unter \texttt{short\_name} kann ein kurzer Name der Applikation angegeben werden. Dieser Name wird verwendet, falls das Endgerät nicht über genügend platz verfügt, um den Originalen Namen anzuzeigen. 

\subsubsection{icons}
Unter \texttt{icons} wird ein Array\footnote{Datentyp, das mehrere Werte speichern kann} mit Bildobjekten gespeichert. Ein Bildobjekt besteht aus einem Dateipfad, unter dem das anzuzeigende Bild gespeichert ist, einer Typ Beschreibung des Bildes zum Beispiel \textit{png} oder \textit{svg}, eine Informationen über die Auflösung des Bildes und optional noch eine Angabe  welchem Zweck das Bild dient. 

Die Gespeicherten Bilder werden als App-Icon auf dem Desktop oder Handy angezeigt. 

\subsubsection{start\_url}
Die angegebene \texttt{start\_url} ist jene \ac{URL} die geöffnet wird, sobald der Nutzer das installierte Icon auswählt und somit die Applikation startet. 
Wird keine explizite Startadresse angegeben, so wird die URL verwendet, von der die PWA installiert wurde. 

\subsubsection{display}

Beim Auswählen des Installierten Icons wird die PWA in einem neuem Fenster geöffnet. Unter \texttt{display} kann angegeben werden, wie das Betriebssystem das Fenster darstellen soll. 
Es kann zwischen \texttt{Fullscreen}, \texttt{Standalone} und \texttt{Minimal User Interface} unterschieden werden. 
Der unterschied zwischen den einzelnen Auswahlmöglichkeiten liegt bei den Navigationselementen, siehe Abbildung \ref{img:Standalone} und \ref{img:minimalui}.

\begin{figure}[!htb]
    \includegraphics[scale=0.3]{Standalone.png}
    \caption{display in Standalone Einstellung}
    \label{img:Standalone}
\end{figure}

\begin{figure}
    \includegraphics[scale=0.3]{minimalui.png}
    \caption{display in Minimal User Interface Einstellung}
    \label{img:minimalui}
\end{figure}

\subsubsection{theme\_color}
Mit Hilfe dieser Einstellung kann die Farbe der oberen Navigationsleiste angepasst werden, wie in Abbildung \ref{img:themeColor} dargestellt ist. Hierbei ist jedoch darauf zu achten, das die Applikation nicht den meta tag theme-color definiert. 

\begin{figure}[!htb]
    \includegraphics[scale=0.3]{themeColor.png}
    \caption{Theme Color auf Weiß geändert}
    \label{img:themeColor}
\end{figure}

\subsubsection{Debugging vom Manifest}

Neben den oben aufgezählten Grundeinstellungen sind viele weitere Möglich. Um nachzuvollziehen, ob alle Einstellungen den Anforderungen entsprechen kann das Manifest mithilfe der Browser Entwicklerwerkzeuge untersucht werden. Unter Google Chrome kann unterm Reiter \textit{Application} das Manifest ausgewählt werden. Darauf hin erhält man folgende Ansicht, siehe Abbildung \ref{img:devManifest}.

\begin{figure}[!htb]
    \centering
    \includegraphics[scale=0.3]{devToolsManifest.png}
    \caption{Entwicklereinstellungen Web-App-Manifest}
    \label{img:devManifest}
\end{figure}


\newpage

\subsection{Service Workers}\label{sec:ServiceWorker}
\begin{quote}
    \textit{Service workers are a fundamental part of a PWA. They enable fast loading (regardless of the network), offline access, push notifications, and other capabilities \cite{Developers2022a}.}
\end{quote}

Der \textit{Service Worker} ist ein wichtiger Grundbaustein um Funktionalitäten wie Push-Notifikations, Hintergrund-Synchronisation und die Möglichkeit, auch Offline die Anwendung auszuführen, zu realisieren. 

Bei dem \textit{Service Worker} handelt es sich um ein script das im Hintergrund des Browsers, unabhängig von der Webanwendung, läuft \cite{Gaunt2021}. Entstanden sind die service worker aus der Verwendung des Application Caches . Die service Worker \ac{API} wächst kontinuierlich und bietet zunehmende weitere Funktionalitäten.

Bei der Verwendung eines Service Worker sollten folgenden Eigenschaften berücksichtigt werden: 
\begin{itemize}
    \item Ein service worker kann zwar nicht direkt das \ac{DOM} einer Seite manipulieren, kann aber auf Requests der Seite mit Responses reagieren und die Seite selbst kann darufhin ihr DOM ändern
    \item Ein service worker ist ein \glqq programmierbarer\grqq{} Proxy, der steuert, wie Requests von der Webseite behandelt werden.
    \item Service workers verwenden die IndexDB API, um client-seitig strukturierte Daten persistent zu speichern.
    \item Service workers verwenden Promises. 
\end{itemize}

\subsubsection{Der Lebenszyklus eines Service Workers}

Der Lebenszyklus eines Service Workers beginnt mit dem registrieren. Die Registrierung erfolgt im JavaScript-Quellcode der PWA. 
Der Service Worker wird vom Browser heruntergeladen und installiert, sobald die PWA mit registrierten Service Worker das erste mal aufgerufen wird. 

Nach einer erfolgreichen Installation wird der Service Worker aktiviert und verwaltet ab dann sämtliche Requests der Applikation. 
Durch den Aktivierten Service Worker kann die Application auch ohne eine bestehende Internetverbindung geladen werden. Der Service Worker verarbeitet den Initialen Request der Applikation, leitet diesen jedoch nicht an den Webserver weiter, siehe Abbildung \ref{img:swonline}, sondern läd die Applikation aus dem Cache Storage, siehe Abbildung \ref{img:swoffline}.

\begin{figure}
    \centering
    \includegraphics[scale=0.45]{swonline.png}
    \caption{Laden einer Applikation ohne Service Worker}
    \label{img:swonline}
\end{figure}

\begin{figure}
    \centering
    \includegraphics[scale=0.45]{swoffline.png}
    \caption{Laden einer Applikation ohne Internetverbindung}
    \label{img:swoffline}
\end{figure}


\subsubsection{Push Notifikations}

Eine Notifikation ist eine Nachricht, die auf dem Endgerät des Nutzers Angezeigt wird. Dies kann als Reaktion auf eine Nutzereingabe geschehen, es kann jedoch auch unabhängig vom Nutzer eine Nachricht von einem Server \glqq gepusht\grqq{} werden und das auch ohne das die Applikation aktiv ist.  Um push Notifikations zu erzeugen werden zwei APIs benötigt. Die \textit{Notifikations API} visualisiert die Nachricht für den Nutzer und die \textit{Push API} erlaubt es den Service Worker, Kapitel \ref{sec:ServiceWorker} Nachrichten zu verwalten, die vom Server gesendet wurden während die Applikation nicht aktiv war. 