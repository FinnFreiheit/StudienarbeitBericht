%!TEX root = ../../main.tex

\chapter{Progressive Web App}

Wie in Kapitel \ref{se:Begriffsklaerung} erwähnt, verfügt eine PWA unter anderem über folgende Funktionen: 
\begin{itemize}
    \item Installierbar,
    \item Zugriff auf Geräteschnittstellen, 
    \item Netzwerkunabhängig,
    \item Push-Notifications.
\end{itemize} 

Im Folgenden werden die theoretischen Grundlagen erläutert, die benötigt werden, um diese Funktionalitäten zu realisieren. 


\section{Web-App-Manifest}\label{sec:webappmanifest}

\begin{quote}
   \textit{The web app manifest is a JSON file that defines how the PWA should be treated as an installed application, including the look and feel and basic behavior within the operating system \cite{Developers2022}. }
\end{quote}

Eine PWA kann auf ein Endgerät wie zum Beispiel ein Desktop oder Handy installiert werden. Um diese Funktion zu realisieren, müssen zusätzliche Informationen wie beispielsweise der Name und das Icon der installierten Applikation in einer Datei festgehalten werden. 

Bei dieser Datei handelt es sich um das sogenannte Web-App-Manifest. Die Informationen sind im \ac{JSON}-Format\footnote{durch Komma getrennte Schlüssel-Wert-Paare} angegeben. 

Ohne das Manifest ist die Applikation nicht installierbar, was die Datei zu einer zwingenden Voraussetzung für eine PWA macht. 
Das Manifest muss mindestens einen \texttt{name}-Schlüssel und einen \texttt{String}-Wert aufweisen. 
Neben dem Namen der Applikation kann ein Manifest über folgende Informationen verfügen: 

\subsection{short\_name}
Unter \texttt{short\_name} kann ein kurzer Name der Applikation angegeben werden. Dieser Name wird verwendet, falls das Endgerät nicht über genügend Platz verfügt, um den originalen Namen anzuzeigen. 

\subsection{icons}
Unter \texttt{icons} wird ein Array\footnote{Datentyp, das mehrere Werte speichern kann} mit Bildobjekten gespeichert. Ein Bildobjekt besteht aus einem Dateipfad, unter dem das anzuzeigende Bild gespeichert ist, einer Typbeschreibung des Bildes, zum Beispiel \textit{png} oder \textit{svg}, einer Informationen über die Auflösung des Bildes sowie optional einer Angabe, welchem Zweck das Bild dient. 

Die gespeicherten Bilder werden als App-Icon auf dem Desktop oder Handy angezeigt. 

\subsection{start\_url}
Die angegebene \texttt{start\_url} ist jene \ac{URL} die geöffnet wird, sobald der Nutzer das installierte Icon auswählt und damit die Applikation startet. 
Wird keine explizite Startadresse angegeben, so wird die URL verwendet, von der die PWA installiert wurde. 

\subsection{display}

Bei Auswählen des installierten Icons wird die PWA in einem neuem Fenster geöffnet. Unter \texttt{display} kann angegeben werden, wie das Betriebssystem das Fenster darstellen soll. 
Es kann zwischen \texttt{Fullscreen}, \texttt{Standalone} und \texttt{Minimal User Interface} unterschieden werden. 
Der Unterschied zwischen den einzelnen Auswahlmöglichkeiten liegt bei den Navigationselementen, siehe Abbildung \ref{img:Standalone} und \ref{img:minimalui}.

\begin{figure}[!htb]
    \includegraphics[scale=0.3]{Standalone.png}
    \caption{display in \glqq Standalone\grqq{}-Einstellung}
    \label{img:Standalone}
\end{figure}

\begin{figure}
    \includegraphics[scale=0.3]{minimalui.png}
    \caption{display in \glqq Minimal User Interface\grqq{}-Einstellung}
    \label{img:minimalui}
\end{figure}

\subsection{theme\_color}
Mit Hilfe dieser Einstellung kann die Farbe der oberen Navigationsleiste angepasst werden, wie in Abbildung \ref{img:themeColor} dargestellt ist. Hierbei ist jedoch darauf zu achten, dass die Applikation nicht den Meta-Tag \texttt{theme-color} definiert. 

\begin{figure}[!htb]
    \includegraphics[scale=0.3]{themeColor.png}
    \caption{Theme Color auf Weiß geändert}
    \label{img:themeColor}
\end{figure}

\subsection{Debugging des Manifests}

Neben den oben aufgezählten Grundeinstellungen sind viele weitere Optionen möglich. Um nachzuvollziehen, ob alle Einstellungen den Anforderungen entsprechen, kann das Manifest mithilfe der Browser-Entwicklerwerkzeuge untersucht werden. Unter Google Chrome kann unter dem Reiter \textit{Application} das Manifest ausgewählt werden. Daraufhin erhält man folgende Ansicht, siehe Abbildung \ref{img:devManifest}.

\begin{figure}[!htb]
    \centering
    \includegraphics[scale=0.3]{devToolsManifest.png}
    \caption{Entwicklereinstellungen Web-App-Manifest}
    \label{img:devManifest}
\end{figure}


\newpage

\section{Service Workers}\label{sec:ServiceWorker}
\begin{quote}
    \textit{Service workers are a fundamental part of a PWA. They enable fast loading (regardless of the network), offline access, push notifications, and other capabilities \cite{Developers2022a}.}
\end{quote}

Der \textit{Service Worker} ist ein wichtiger Grundbaustein, um Funktionalitäten wie Push-Notifications und Hintergrund-Synchronisation zu realisieren sowie die Möglichkeit, die Anwendung auch offline auszuführen. 

Bei einem \textit{Service Worker} handelt es sich um ein Script, das im Hintergrund des Browsers - unabhängig von der Webanwendung - läuft \cite{Gaunt2021}. Entstanden sind die Service Workers aus der Verwendung des Application Caches. Die Service Worker \ac{API} wächst kontinuierlich und bietet zunehmend weitere Funktionalitäten.

Bei der Verwendung eines Service Workers sollten folgenden Eigenschaften berücksichtigt werden: 
\begin{itemize}
    \item Ein Service Worker kann zwar nicht direkt das \ac{DOM} einer Seite manipulieren, kann aber auf Requests der Seite mit Responses reagieren. Die Seite selbst kann darufhin ihr DOM ändern.
    \item Ein Service Worker ist ein \glqq programmierbarer\grqq{} Proxy, der steuert, wie Requests von der Webseite behandelt werden.
    \item Service Workers verwenden die IndexDB-API, um client-seitig strukturierte Daten persistent zu speichern.
    \item Service Workers verwenden Promises. 
\end{itemize}

\subsection{Der Lebenszyklus eines Service Workers}

 
Der Service Worker wird von einem Browser in einem eigenen Thread unabhängig von der Applikation ausgeführt. Dementsprechend besitzt der Service Worker einen eigenen Lebenszyklus, der im Folgenden beschrieben wird. 

\subsubsection{Registrieren}
Der Lebenszyklus eines Service Workers beginnt mit der Registrierung. Dies erfolgt mithilfe der \texttt{register()}-Methode in der \texttt{serviceWorker}-Eigenschaft des \texttt{navigator}-Objektes \cite{Navigator2022}, siehe Listing \ref{lst:serviceWorker} Zeile 2. Um dem Konzept des Progressive Enhancements zu folgen (Abschitt \ref{se:Begriffsklaerung}), sollte vor der Registrierung überprüft werden, ob die Service-Worker-Schnittstellen vorhanden sind. Hierfür wird die \texttt{serviceWorker}-Eigenschaft innerhalb des \texttt{navigator}-Objekts abgefragt, siehe Listing \ref{lst:serviceWorker} Zeile 1. 

\begin{lstlisting}[caption = Registrierung des Service Workers, label = lst:serviceWorker, float = !htb]
    if('serviceWorker' in navigator){
        navigator.serviceWorker.register('./sw.js')
        .then(registration => console.log(registration))
        .catch(error => console.error(error));
    }
\end{lstlisting}

Das Aufrufen der \texttt{register()}-Methode gibt einen Promise zurück, welcher in Zeile 3 und 4 behandelt wird. 
Im Erfolgsfall (then-Methode) wird ein Objekt des Typs \texttt{ServiceWorkerRegistration} übergeben. Das Objekt verfügt unter anderem über folgende Schnittstellen: 
\begin{itemize}
    \item mit der \texttt{update()}-Methode wird die Aktualisierung des Service Workers ausgelöst,
    \item die \texttt{unregister()}-Methode hebt die Registrierung des Service Workers auf, welches im Anschluss durch den Browser gelöscht wird,
    \item die Eingenschaft \texttt{updatefound} wird aufgerufen, sobald der Browser eine neue Version des Service Workers gefunden hat. 
\end{itemize}

\subsubsection{Installieren}

Mit der Registrierung erfolgt ebenfalls die Installation des Service-Worker-Skripts, welches bei der Registrierung übergeben wurde, siehe Listing \ref{lst:serviceWorker} Zeile 2.
In der Installationsphase lädt der Service Worker Ressourcen von dem Webserver und speichert diese lokal ab. Wichtig ist hierbei, dass die Installationsphase erst dann beendet werden darf, wenn alle benötigten Ressourcen heruntergeladen wurden. Um dies zu realisieren, steht die \texttt{waitUntil()}-Methode zur Verfügung, die eine Promise-Kette abarbeitet. Diese Methode öffnet den zugehörigen Cache. Im Cache werden durch die \texttt{addAll()}-Methode alle benötigten Ressourcen gespeichert, die normalerweise bei dem Öffnen der Applikation durch den Webserver angefordert werden. Dies ermöglicht, dass die Applikation in Zukunft auch ohne Netzwerkverbindung aus dem lokalen Cache geladen werden kann. 



Die im Cache gespeicherten Dateien können mithilfe der Google Chrome Developer Tools unter dem Reiter \textit{Application} eingesehen werden. 

\subsubsection{Aktivieren}

Nach der Installation geht der Service Worker in den aktiven Zustand über. Ein aktivierter Service Worker ist in der Lage, funktionale Ereignisse zu behandeln wie zum Beispiel das Beantworten von HTTP(S) Anfragen oder das Entgegennehmen von Pushbenachrichtigungen. 
Der Service Worker übernimmt somit die Kontrolle der Applikation. Dies geschieht jedoch erst, wenn ein neuer Browser-Kontext geöffnet wird. Aktiviert man Service Worker während der Laufzeit der Webanwendung, kontrolliert er alle ihrer HTTP(S)-Anfragen, darunter auch jene, deren Ressourcen noch nicht im Cache gespeichert wurden.

\subsubsection{Überschreiben}

Sobald ein neuer Service Worker für einen bestimmten Scope aktiviert wurde, terminiert der Webbrowser das alte Skript. Der alte Service Worker geht in den Zustand \textit{redundant} und wird in dem weiteren Verlauf entfernt. 

Unter Apple Safari wird ein registrierter Service Worker automatisch deinstalliert, sobald die Webseite des Service Workers mehrere Wochen nicht aufgerufen wurde. 


\section{Push-Notifications}

Eine Notification ist eine Nachricht, die auf dem Endgerät des Nutzers angezeigt wird. Dies kann als Reaktion auf eine Nutzereingabe geschehen. Es kann jedoch auch unabhängig von dem Nutzer eine Nachricht von einem Server \glqq gepusht\grqq{} werden. Die ist auch ohne das Aktivsein der Applikation möglich.  Um Push-Notifications zu erzeugen, werden zwei APIs benötigt. Die \textit{Notifications API} visualisiert die Nachricht für den Nutzer. Die \textit{Push API} erlaubt es wiederum dem Service Worker (Kapitel \ref{sec:ServiceWorker}) Nachrichten zu verwalten, die durch den Server gesendet wurden, während die Applikation inaktiv war. 

Neben Frontend, Backend und Service Worker sind an der Push-Kommunikation zudem der Push-Service sowie der Push-Dienst beteiligt. Jeder Browserhersteller hat einen eigenen Push-Dienst (eng. unser agent) implementiert. Google Chrome verwendet \textit{Firebase Cloud Messaging}, Mozilla Firefox den \textit{Mozilla Push Service} und Microsoft Edge die \textit{Windows Push Notification Services}. Der Push-Dienst nimmt die Pushnachrichten des Push-Services entgegen und leitet diese an den Service Worker weiter. Die Kommunikation mit den Push-Diensten ist durch die \ac{IETF} spezifiziert \cite{rfc8030}. Somit müssen keine gesonderten Implementierungen für die einzelnen Push-Dienste vorgenommen werden. 

Wie die einzelnen Kommunikationspartner im Genauen miteinander interagieren ist in dem Sequenzdiagramm in Abbildung \ref{img:sequenceDiagram} dargestellt und wird im Folgendem erläutert. 

\newpage
\begin{figure}[!htb]
    \centering
    \includegraphics{Push_sequence_diagram.png}
    \caption{Sequenzdiagramm für die Ereignisse einer Anmeldung, eines Push-Nachrichttransfers und einer Abmeldung \cite{PushW3}}
    \label{img:sequenceDiagram}
\end{figure}
\newpage


\subsection{Push-Registrierung}\label{sec:pushRegistrirung}

Um Push-Nachrichten empfangen zu können, muss die Webanwendung über den Webbrowser eine Push-Registrierung bei dem Push-Dienst beantragen. Diese Beantragung erfolgt über die PushAPI, die über den ServiceWorker - genauer die \texttt{ServiceWorkerRegistration} - zur Verfügung gestellt wird. Aus diesem Grund ist es zwingend notwendig, dass der Service Worker registriert ist. 
Bei der Registrierung sollte das Konzept des \textit{Progressive Enhancements} berücksichtigt werden, indem die Browserkompatibilität vor der Registrierung abgefragt wird, siehe Listing \ref{lst:regPush}. 

\begin{lstlisting}[caption={Push-Registrierung unter Berücksichtigung der Konzepte des Progressive Enhancements}, label={lst:regPush}, float={!htb}]
    if('serviceWorker' in navigator){
        navigator.serviceWorker.register('sw.js');
        navigator.serviceWorker.ready.then(registration => {
            if('PushManager' in window){
                registerForPush(registration.pushManager);
            }
        });
    }
\end{lstlisting}



Falls der Browser über die nötigen Funktionen verfügt, kann die \texttt{subscribe()}-Methode der \texttt{ServiceWorkerRegistration} aufgerufen werden, welches ein Objekt des Typs \texttt{PushSubscription} zurückgibt. 
Die PushSubscription verfügt über ein Konfigurationsobjekt mit zwei Eigenschaften. Die erste Eigenschaft trägt den Namen \texttt{userVisibleOnly} und ist eine boolsche Eigenschaft, die  beschreibt, ob Pushnachrichten immer zu einer für den Anwender sichtbaren Meldung führen. Diese hat standardmäßig den Wert \texttt{false}. Innerhalb des Browsers Google Chrome führt das jedoch zu Komplikationen, da der Browser aus Gründen der Datensicherheit keine sogenannte \textit{Silent Push} Nachrichten zulässt. Daher sollte diese Eigenschaft auf den Wert \texttt{true} geändert werden. 

Die zweite Eigenschaft des Konfigurationsobjekts bezeichnet man als \texttt{applicationServerKey}. Diese Eigenschaft wird benötigt, um die Nachrichten zwischen dem Push-Dienst und der Anwendung zu verschlüsseln. Zur Verschlüsselung wird das \ac{VAPID}-Protokoll verwendet, das von der IETF spezifiziert wurde \cite{rfc8292}. Die Verschlüsselung benötigt einen öffentlichen und einen privaten Schlüssel. Der öffentliche Schlüssel ist unter der \texttt{applicationServerKey}-Eigenschaft gespeichert. Der private Schlüssel wird wiederum in dem Backend der Anwendung hinterlegt. 

Um die Registrierung abzuschließen, muss der Anwender sein Einverständnis zum Empfangen von Push-Benachrichtigungen geben. Für die Abfrage öffnet der Browser einmalig ein Dialogfenster. Wird kein Einverständnis vom Nutzer erteilt, so wird auf diesem speziellen System, in keinem Fall, erneut danach gefragt. Der Anwender müsste die Berechtigung manuell über das zugehörige Webseitenmenü erteilen, sollte er sich umentscheiden. 

Nachdem die Berechtigung durch den Nutzer erteilt wurde, ist die Registrierung abgeschlossen. Die Anwendung ist nun in der Lage Push-Nachrichten zu empfangen. 

\subsection{Informationsaustausch}

Nach einer erfolgreichen Registrierung erhält der Anwender ein Subscription-Objekt vom Typ \texttt{PushSubscription}.
Das Subscription-Objekt wird dem Backend übergeben. Jeder Anwender, der die Pushbenachrichtigung abonniert hat, besitzt ein eigenes Subscription-Objekt, welches im Backend der Anwendung verwaltet wird. 
Die Subscription-Objekte sind für jeden Anwender einzigartig, da das Subscription-Objekt über die Eigenschaft \texttt{endpoint} verfügt. Hinter dieser Eigenschaft, befindet sich die URL des API-Endpunktes vom Push-Dienst. Der Browserspezifische Push-Dienst versendet die Push-Nachricht an der Service Worker und somit an den Anwender. 

Die API-Endpunkte der Push-Dienste sind standardisiert. Über den Endpunkt kann nun eine Benachrichtigungen durch das Backend angestoßen werden. Hierbei empfehlt es sich eine Web-Push-Bibliothek zu verwenden, um die standardisierte Kommunikation über den Endpunkt mit dem Push-Dienst zu erleichtern. 

\subsection{Zusammenfassung Push-Notifikation}

Im folgenden Szenario werden die einzelnen Schritte zusammengefasst, um eine Push-Nachricht zu erhalten. Hierfür wird von einer Zeitschrift-Webseite ausgegangen. Der Nutzer besucht die Webseite zum ersten mal über den Browser \textit{Chrome}. Automatisch wird der Service Worker im Hintergrund installiert. Der Nutzer wird nach der Einwilligung zum erhalten von Push-Nachrichten gefragt, und bestätigt diese. Durch die Bestätigung, wird ein Objekt vom Typ \texttt{PushSubscription} erzeugt und an das Backend der Zeitschrift-Webseite gesendet. Das \texttt{PushSubscription} Objekt verfügt über die URL eines API-Endpunktes. Der Endpunkt ermöglicht die Kommunikation zum Push-Dienst von Google mit dem Namen \textit{Firebase Cloud Messaging}. 
Im Backend wird das \texttt{PushSubscription}-Objekt in einer Datenbank abgelegt. 

Die Zeitschrift-Firma möchte nun alle Abonnenten darüber informieren, das sie einen neuen Artikel erstellt haben. Dafür wird im Backend über alle in der Datenbank gespeicherten \texttt{PushSubscripten}-Objekte iteriert. Über das \texttt{Web-Push-Protokoll} und dem im \texttt{PushSubscription}-Objekt hinterlegten Endpunkten wird die Nachricht, das ein neuer Artikel hochgeladen wurde an die Push-Dienste gesendet. 

Die Push-Dienste leiten diese Nachricht an den Nutzer über den Service Worker weiter oder speichern diese, falls der Nutzer nicht erreichbar ist. 







\section{Progressive Web Apps in Angular}

Angular (Abschnitt \ref{Angular}) ist geeignet, um eine PWA zu entwickeln, da sowohl das Framework als auch die Progressive Web Apps Entwicklungen von Google sind. 
Um aus einer Angular-Anwendung eine PWA zu erstellen, muss das Paket \texttt{@angular/pwa} über das \ac{CLI} hinzugefügt werden.
Durch das Hinzufügen des Pakets werden im Hintergrund folgende Veränderungen an dem Projekt durchgeführt: 
\begin{itemize}
    \item Das Paket \texttt{@angular/service-worker} wird zu dem Projekt hinzugefügt,
    \item der Build Support für den Service Worker wird in der Angular CLI aktiviert,
    \item das \texttt{ServiceWorkerModule} wird in dem \texttt{AppModule} importiert,
    \item die Datei index.html wird um einen Link zu dem Web App Manifest (Abschitt \ref{sec:webappmanifest}) und um relevante Meta-Tags ergänzt,
    \item Icon-Dateien werden erzeugt und verlinkt,
    \item die Konfigurationsdatei \textit{ngsw-config.json} für den Service Worker wird erzeugt. 
\end{itemize}

Im Anschluss kann die Applikation alle Grundfunktionen einer PWA ausführen. Die Anwendung kann installiert werden. Innerhalb des Web App Manifests steht der Projektname und das unter \texttt{assets} abgespeicherte Angular-Icon wird als App-Icon verwendet. Nach dem erstmaligen Öffnen der Applikation in dem Browser wird der Service Worker installiert. 

Angular unterstützt den Entwickler beim Cachen von Daten und Dateien durch eine Konfigurationsdatei mit dem Namen \texttt{ngsw-config.json}. Die Grundeinstellung wird in Listing \ref{lst:ngswConfig} beschrieben. Unter \texttt{assetGroups} werden Ressourcen angegeben, die teil der Angular-Anwendung sind und im Cache gespeichert werden. Aus der Datei kann entnommen werden, das die \texttt{index.html}, das webmanifest, alle css und JavaScript-Dateien sowie alle Inhalte aus dem \textit{assets}-Verzeichnis im Cache gespeichert werden. 
Somit kann die Applikation auch ohne Netzwerkverbindung gestartet werden, da alle benötigten Ressourcen im Cache hinterlegt sind. 

Zusätzlich zu den Ressourcen der Angular-Anwendung können auch 
die Antworten von API-Anfragen im Cache gespeichert werden.
Hierfür werden unter der Eigenschaft \texttt{dataGroups} URL-Pfade von API-Endpunkten angegeben. Sobald der Service Worker eine API-Anfrage des Nutzer weiterleitet, der mit einem URL-Pfad aus \texttt{dataGroups} übereinstimmt, wird die Antwort im Cache gespeichert. 
Somit ist der Service Worker in der Lage auch ohne Netzwerkverbindung die API-Anfrage aus dem Cache zu beantworten. Hierbei kann zwischen zwei Cache-Strategien unterschieden werden.
Bei der \texttt{freshness}-Strategie wird jede API-Anfrage an den Entsprechenden Endpunkt weitergeleitet. 
Die Daten werden erst dann aus dem Cache bezogen, wenn der Endpunkt nicht erreichbar ist. Diese Strategie wird gewählt, wenn sich die Angefragten Daten regelmäßig ändern. 
Bei der \texttt{performance}-Strategie werden die Daten immer aus dem Cache geladen. Die Daten aus dem Cache werden in regelmäßigen abständen aktualisiert. Das Laden aus dem Cache ist deutlich schneller als ein Netzwerkzugriff. Diese Strategie eignet sich besonders für Daten die nicht oft geändert werden. 

\begin{lstlisting}[caption={Angular \textit{ngsw-config.json}-Datei zur Angabe der Ressourcen, die durch den Service Worker in den Cache gespeichert werden sollen}, label=lst:ngswConfig, float=!htb ]
    "assetGroups": [
        {
          "name": "app",
          "installMode": "prefetch",
          "resources": {
            "files": [
              "/favicon.ico",
              "/index.html",
              "/manifest.webmanifest",
              "/*.css",
              "/*.js"
            ]
          }
        },
        {
          "name": "assets",
          "installMode": "lazy",
          "updateMode": "prefetch",
          "resources": {
            "files": [
              "/assets/**",
              "/*.(eot|svg|cur|jpg|png|webp|gif|otf|ttf|woff|woff2|ani)"
            ] 
\end{lstlisting}






\section{Endgerät-Emulation}

Um eine PWA Lokal auf Endgeräten zu testen, können diese emuliert werden.   

\subsection{Android}
Um ein Android-Handy zu emulieren, kann das Programm \textit{Android Studio} verwendet werden. In Android Studio kann mithilfe des \ac{AVD Managers} ein virtuelles Handy gestartet werden, Abbildung \ref{Android1}. In dem emuliertem Handy kann der Browser geöffnet werden. Um auf die lokale IP-Adresse des Rechners zugreifen zu können, muss innerhalb des Browsers auf dem Handy die IP-Adresse \texttt{10.0.2.2:8080} eingegeben werden. 

\begin{figure}[!htb]
    \centering
    \includegraphics[scale=0.4]{emuliertHandy.png}
    \caption {Emuliertes Android Handy in Android Studio}
    \label{Android1}
\end{figure}

\subsection{IOS}

Um ein IOS-Gerät zu emulieren, wird das Programm XCode benötigt, welches jedoch nur auf einem MacOS Gerät installiert werden kann. In XCode kann ein einfaches \texttt{HelloWorld} IOS-Projekt erstellt werden. Wird dieses Ausgeführt öffnet sich automatisch ein emuliertes iPhone, siehe Abbildung \ref{img:emuIOS}.

\begin{figure}
    \centering
    \includegraphics[scale=0.4]{emulirtesIOS.png}
    \caption{Emuliertes IPhone in XCode}
    \label{img:emuIOS}
\end{figure}

\subsection{ngrok}


ToDO 


