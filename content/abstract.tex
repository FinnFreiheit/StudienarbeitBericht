%!TEX root = ../main.tex

\pagestyle{empty}

% override abstract headline
\renewcommand{\abstractname}{Abstract}

\begin{abstract}

    Eine Progressive Web App (PWA) ist eine Website, welche eine Vielzahl an Charakteristika nativer Apps aufweist. Wie jede andere Website auch können PWAs mit JavaScript, HTML und CSS erstellt werden. Was sie in ihrer Nutzung jedoch von herkömmlichen responsiven Webseiten unterscheidet, sind app-spezifische Merkmale wie die Installierbarkeit, der Zugriff auf Geräteschnittstellen, die Verfügbarkeit im Offline-Modus sowie die Möglichkeit der Aktivierung von Push-Nachrichten.

    Dennoch können PWAs auch gegenüber nativen Apps einen entscheidenden Vorteil liefern. PWAs sind unabhängig von Plattformen. App-Entwickler stehen oft vor der Herausforderung, eine App in mehreren Programmiersprachen zu entwickeln, um sie für verschiedene Betriebssysteme und Geräte zur Verfügung stellen zu können. 
    
    Der daraus entstehende Aufwand soll mithilfe von PWAs minimiert werden. Eine Entwicklung in verschiedenen Programmiersprachen ist hierbei nicht nötig; die Verwendung von Website-Entwicklungs-Tools und -Bibliotheken ist ausreichend. Die Aufwandsminimierung soll dennoch nicht mit einer Einschränkung der vollumfänglichen Einsetzbarkeit auf allen Plattformen einhergehen. 
    
    Ob PWAs plattformübergreifend alle Ansprüche erfüllen können und reif genug sind, eine relevante Alternative zu nativen Apps darzustellen, soll in der Arbeit ermittelt werden.  

\end{abstract}